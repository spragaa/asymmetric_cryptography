\documentclass{article}
\usepackage{graphicx}
\usepackage[english,ukrainian]{babel}
\usepackage[letterpaper,top=2cm,bottom=2cm,left=3cm,right=3cm,marginparwidth=1.75cm]{geometry}
\usepackage{amsmath, graphicx, booktabs, listings, xcolor, tcolorbox, lipsum, siunitx, multirow, hyperref, pgfplots, inputenc}

\title{Вивчення криптосистеми Рабіна та протоколів доведення з нульовим розголошенням}
\date{}

\begin{document}

\maketitle

\section{Мета}
Ознайомлення із криптосистемою Рабіна та особливостям її реалізації. Ознайомлення з криптографічними протоколами взагалі та протоколами доведення знання без розголошення зокрема.Ознайомлення із перевагами, недоліками та особливостями реалізації різних криптографічних протоколів. Аналіз наведеного протоколу;реалізація атаки на цей протокол.

\section{Постановка задачі}
Створити реалізацію:
\begin{itemize}
    \item Криптосистеми Рабіна з функціями шифрування/дешифрування
    \item Протоколу доведення з нульовим розголошенням для квадратного кореня
    \item Атаки на криптосистему Рабіна через знаходження різних квадратних коренів
\end{itemize}

\section{Хід роботи}

\subsection{Компоненти реалізації}
\begin{itemize}
    \item Генерація простих чисел Блюма (p,q ≡ 3 (mod 4))
    \item Шифрування повідомлень за схемою Рабіна
    \item Протокол доведення знання квадратного кореня
    \item Атака через пошук різних квадратних коренів
\end{itemize}

\subsection{Криптографічні операції}
\begin{itemize}
    \item Обчислення квадратичного лишку за модулем n
    \item Знаходження квадратних коренів за модулем n
    \item Перевірка чисел Блюма
    \item Реалізація протоколу доведення з нульовим розголошенням
\end{itemize}

\section{Результати тестування}

\subsection{Початкові параметри системи}
\begin{verbatim}
Згенеровані ключі:
Приватний ключ P: 
106423144462122888331345348821960685812431668195277488806264724450150352770927

Приватний ключ Q: 
70019178177211439058777081598650860080965790003643145181032391666254511915867

Публічний ключ N: 
745166111427249535523951284070617904377551442042308879298076745916849035275213
182718995796042494690512037473206078590937260308264131164631431654294759870

Публічний ключ B: 
293912637758114797489162356425548013444482194839879543905747609429957386168706
779668135607416977831596614803588781728132757929258184407162546231130518502
\end{verbatim}

\subsection{Ключова інформація користувачів}
\subsubsection{Ключі Аліси}
\begin{verbatim}
Публічний ключ N:
47B6B6BD98CDD1D6655FC4FD6BE1CCDCAD902DC653F9EE8D7A251C927C85D15D
1A13ED64C82BA8BE3670B67A0E7EAB8B49459EDCA31D575065190F0A44C7D989

Публічний ключ B:
145D9BCA8B59E46B7F9A895BC97889ECFF0B4DB38464FC017F2DF392B9D7258A
670F2E18DA5771F49DB702FDE82C651D8CA3D946F1DCB1E276E25B946B3E4606

Приватний ключ P:
5DACB00A42D3DED1A74B0A4F08F22B936E0D94418EC5A3A2A333B6D840E6633B

Приватний ключ Q:
C3FBE67853FBEE23BEFAD8667380BEE8A5C24700F0B5F95EF6CDCBC6025F620B
\end{verbatim}

\subsubsection{Ключі Боба}
\begin{verbatim}
Публічний ключ N:
8E46F5F7AABB1525BE51143BE01DFC032022D58D733567A708DA0DE968CAA63E
E71AFDFE9B3BACC362998A898882386E8260FD166EE55A582EB3B0140D23A175

Публічний ключ B:
381E2685A03BDFF452359940DC154446091DBF8E8C70AD713F5536281F619F12
8DBB2621A9E136DDCF360F26A96549CF1F9F890FA3DB16419863ED41AEF986FE

Приватний ключ P:
EB495D7C24DEA1061ED3FE1EF53A67558B31D5DC4822FC872FE2B62B0821576F

Приватний ключ Q:
9ACD734F9C232750A5A0467D7AB1CB7AFB39C9A265FAC001141BE33B1F50C35B
\end{verbatim}

\subsection{Тестування обміну повідомленнями}
\begin{verbatim}
Оригінальне повідомлення Аліси: 
48656C6C6F20426F6221

Зашифроване повідомлення від Аліси до Боба:
42E993D6337F87F33D3D45514EFBAEFC47B62F3D8A9FFBEB6EAB485E04F887ED
96DB061BB7BA78D3F223213C48B2A2826B0237595DBA8576FDB1F5433097C143

Індикатори шифрування: (1, 1)

Розшифроване повідомлення Бобом:
48656C6C6F20426F6221
\end{verbatim}

\subsection{Тестування підписаного повідомлення}
\begin{verbatim}
Оригінальне секретне повідомлення:
5365637265742066726F6D20416C69636521

Підпис Аліси:
1762EE5B37C80A683FF70D695E55464039EFA2F02B25102A5E46B70E119ADFD7
93C8D6808FE45DCCF2AEF73FDE72B16BE1469C8AD0B63BE8190C387EEC01B187
\end{verbatim}

\subsection{Тестування обміну ключами}
\begin{verbatim}
Сесійний ключ: DEADBEEF
\end{verbatim}

\section{Висновок}
У роботі було реалізовано криптосистему Рабіна та протокол доведення з нульовим розголошенням. Продемонстровано вразливість системи до атаки через знаходження різних квадратних коренів, що дозволяє факторизувати модуль. Підтверджено важливість використання чисел Блюма для коректної роботи системи.

\end{document}