\documentclass[12pt]{article}
\usepackage[utf8]{inputenc}
\usepackage{amsmath}
\usepackage{graphicx}
\usepackage{booktabs}
\usepackage{float}
\usepackage[ukrainian]{babel}

\title{\textbf{Реалiзацiя атаки Хеллмана на геш-функцiї}}
\author{}
\date{}

\begin{document}
\maketitle

\section{Опис Атаки}
Атака Хеллмана з компромісом часу-пам'яті - це криптографічна атака, яка намагається знайти прообрази геш-функцій шляхом балансування вимог до обчислювального часу та пам'яті. Атака складається з двох фаз:

\subsection{Передобчислення}
\begin{itemize}
  \item генерація ланцюжків геш-значень з використанням функції редукції
  \item зберігання лише початкових та кінцевих точок ланцюжків у таблицях
\end{itemize}

\subsection{Онлайн фаза}
\begin{itemize}
  \item при заданому геш-значенні, спробувати знайти його прообраз використовуючи збережені ланцюжки
  \item перевірка на існування отриманого значення як кінцевої точка в таблицях
\end{itemize}

Згідно з теоремою Хеллмана, ймовірність успіху P приблизно дорівнює:
$ P \approx 1 - e^{-KL/N} $, де N - розмір простору виходів геш-функції.

\section{Деталі Реалізації}

\subsection{Деталі реалізації}
\begin{itemize}
  \item використана геш-функція Whirlpool
  \item реалізовано паралельну генерацію ланцюжків з використанням Rayon
\end{itemize}

\section{Приклад атаки}

Використані параметри:
\begin{itemize}
  \item розмір гешу: 32 біти
  \item довжина ланцюжка (L): 1024
  \item кількість ланцюжків (K): 1048576
  \item кількість таблиць: 4
\end{itemize}

Приклад виконання:
\begin{verbatim}
Вхідне повідомлення (256 біт): 
[73, 26, 95, 105, 63, 238, 78, 37, 243, 124, 140, 128, 
44, 245, 219, 193, 168, 91, 176, 48, 142, 23, 255, 153, 
68, 21, 102, 69, 34, 39, 243, 212]

Геш-значення: [100, 50, 140, 148]
Знайдений прототип: [68, 114, 231, 158]
Кількість спроб: 1494
\end{verbatim}

\section{Результати експерименту}

\begin{table}[H]
\centering
\begin{tabular}{@{}c|ccc@{}}
\multicolumn{4}{c}{} \\
\midrule
K/L & 2^{10} & 2^{11} & 2^{12} \\
\midrule
2^{20} & 99.10\% & 99.80\% & 100.00\% \\
2^{22} & 99.80\% & 100.00\% & 100.00\% \\
2^{24} & 100.00\% & 100.00\% & 100.00\% \\
\bottomrule
\end{tabular}
\caption{Ймовірність успіху (одна таблиця)}
\end{table}

\begin{table}[H]
\centering
\begin{tabular}{@{}c|ccc@{}}
\multicolumn{4}{c}{} \\
\midrule
K/L & 2^{10} & 2^{11} & 2^{12} \\
\midrule
2^{20} & 98.40\% & 99.30\% & 99.90\% \\
2^{22} & 99.60\% & 99.90\% & 100.00\% \\
2^{24} & 100.00\% & 100.00\% & 100.00\% \\
\bottomrule
\end{tabular}
\caption{Ймовірність успіху (декілька таблиць)}
\end{table}

\begin{table}[H]
\centering
\begin{tabular}{@{}c|ccc@{}}
\multicolumn{4}{c}{} \\
\midrule
K/L & 2^{10} & 2^{11} & 2^{12} \\
\midrule
2^{20} & 87.63 & 186.04 & 378.86 \\
2^{22} & 381.65 & 770.83 & 1538.72 \\
2^{24} & 1537.90 & 3067.28 & 6126.09 \\
\bottomrule
\end{tabular}
\caption{Час виконання (секунди)}
\end{table}

\subsection{Порівняння з теоретичними оцінками}
Експериментальні результати відповідають теоретичним прогнозам, з показниками успішності, що наближаються до 100\% для більших значень параметрів. Варіант з кількома таблицями показує кращі показники успішності за рахунок збільшення використаної пам'яті.

\section{Висновок}
Результати підтверджують, що атака Хеллмана з компромісом час-пам'ять залишається практичним підходом для знаходження прообразів хеш-функцій, особливо коли простір виходів обмежений.

\end{document}