\documentclass[12pt]{article}
\usepackage[utf8]{inputenc}
\usepackage{amsmath}
\usepackage{graphicx}
\usepackage{booktabs}
\usepackage{float}
\usepackage[ukrainian]{babel}

\title{\textbf{Реалiзацiя криптосистеми RSA}}
\author{}
\date{}

\begin{document}
\maketitle

\section{Вступ}
RSA - це асиметрична криптосистема, що використовується для шифрування та цифрового підпису. Алгоритм базується на складності факторизації добутку двох великих простих чисел.

\subsection{Основні компоненти}
\begin{itemize}
  \item Генерація ключів (публічного та приватного)
  \item Шифрування повідомлень
  \item Цифровий підпис
  \item Обмін ключами
\end{itemize}

\subsection{Принципи RSA}
RSA базується на наступних принципах:
\begin{itemize}
  \item Вибір двох великих простих чисел p і q
  \item Обчислення модуля \(n = p * q\)
  \item Обчислення функції Ейлера \(φ(n) = (p-1)(q-1)\)
  \item Вибір відкритої експоненти e
  \item Обчислення секретної експоненти d
\end{itemize}

\section{Приклад роботи}

Використані параметри:
\begin{itemize}
  \item Розмір ключа: 256 біт
  \item Значення публічної експоненти (E): 0x10001
\end{itemize}

\subsection{Ключі Аліси}
\begin{verbatim}
Публічна експонента (E):
10001

Публічний модуль (N):
1BE32B781E2188EC9A83F61062AF5695535732EF2F628B9E25001C7FAE4A5B7B
47EE78EC6FE22E9E450166297B51A2DE88109D0C9C682DCB4768AE89FADEDF49

Приватний ключ (D):
A9108E77A833A5E6C2D940EA155CE78C61B44315CC2FA23F1E118EB481EE48A8
26AE77CCA86B5E18C76A643F045CDCD385B906E19E3906531BB3068553B801
\end{verbatim}

\subsection{Ключі Боба}
\begin{verbatim}
Публічна експонента (E):
10001

Публічний модуль (N):
5CBDEFD0905EBFBE3DBB5F0C25629991DFA666732B97ED71FA2BA5AC71509633
F81E709FE3C402F5B71072B331BA7A54A7CBC6409C37326623B23C716EFA3805

Приватний ключ (D):
2F16D1B95854838F18D116859845E42BC91F384DAA08CE2B5D16E45C167D1086
54CAC09D634E31403107F44773567F3FB233DFD8AACF5474E6D8B667F4066E01
\end{verbatim}

\section{Тестування функціональності}

\subsection{Обмін повідомленнями}
\begin{verbatim}
Оригінальне повідомлення: 48656C6C6F20426F6221
Зашифроване повідомлення: 
5692720C48935A47A2E1EF8AFE46138BDDA24D6540F9D6498D6ECF7B033DFF1A
D12CB01FE943A141A95B187FB42B97DF0D57922891D9E748A02F43274083CAEE
Розшифроване повідомлення: 48656C6C6F20426F6221
\end{verbatim}

\subsection{Тест цифрового підпису}
\begin{verbatim}
Секретне повідомлення: 
5365637265742066726F6D20416C69636521

Підпис Аліси:
13E3F8DDB6640592B41FF233F3FC7A91871B47FEC7079117630AEC531516F53C
09AD5F32E85095F1848B2251803FD3B7A0EAEAB921977301AB7CE6A0A05E59DE

Зашифроване повідомлення:
B6DB06EF2396F8CACC3303E0D026120079320BC972E3CFC98F357AC9BC711C7B
179C21266C60544756F18A8E2578734FD1062A03C13E73B33F9C6CEDB2C675C

Результат верифікації: Успішно
\end{verbatim}

\subsection{Обмін ключами}
\begin{verbatim}
Сесійний ключ: DEADBEEF
Зашифрований ключ:
1fc8242f1744d07e10b946065135e8f6890587e3b979f49cfc5bb8bc18be2a47
ebb64ec6babdea0552ef3bb9bf2c16e9e4b065f03089dcbb5d96e5e80696ffb4

Зашифрований підпис:
3a31089c88ce4f6be4384759612aa882b4e824e1e2200a27b64f452c6efe82d3
691cbd86d0a24526a20aeb5936a3c33d45ab6981cbb42aef1034b153ee811f21

Результат обміну: Успішно
\end{verbatim}

\section{Висновок}
Реалізована RSA криптосистема демонструє правильне  виконання основних криптографічних операцій, включаючи шифрування, цифровий підпис та обмін ключами.

\end{document}