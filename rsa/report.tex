\documentclass{article}
\usepackage{graphicx}
\usepackage[english,ukrainian]{babel}
\usepackage[letterpaper,top=2cm,bottom=2cm,left=3cm,right=3cm,marginparwidth=1.75cm]{geometry}
\usepackage{amsmath, graphicx, booktabs, listings, xcolor, tcolorbox, lipsum, siunitx, multirow, hyperref, pgfplots, inputenc}

\title{Вивчення криптосистеми RSA та алгоритму електронного підпису; ознайомлення з методами генерації параметрів для асиметричних криптосистем}
\date{}

\begin{document}

\maketitle

\section{Мета}
\quad Ознайомлення з тестами перевірки чисел на простоту і методами генерації ключів для асиметричної криптосистеми типу RSA; практичне ознайомлення з системою захисту інформації на основі криптосхеми RSA, організація з використанням цієї системи засекреченого зв'язку й електронного підпису, вивчення протоколу розсилання ключів.

\section{Постановка задачі}
\quad Створити повноцінну реалізацію криптосистеми RSA з наступними можливостями:
\begin{itemize}
    \item Генерація ключової пари (публічний та приватний ключі)
    \item Шифрування та дешифрування повідомлень
    \item Створення та перевірка цифрового підпису
    \item Безпечний обмін ключами
\end{itemize}

\section{Хід роботи}
\quad 
Програма реалізує криптосистему RSA з використанням наступних компонентів:

\subsection{Генерація простих чисел}
\begin{itemize}
    \item Використання генератора BBS для генерації випадкових чисел
    \item Попередня перевірка методом пробних ділень
    \item Тест Міллера-Рабіна для перевірки простоти числа
\end{itemize}

\subsection{Криптографічні операції}
\begin{itemize}
    \item Модульне піднесення до степеня для шифрування/дешифрування
    \item Обчислення мультиплікативного оберненого за модулем для генерації приватного ключа
    \item Реалізація цифрового підпису на основі RSA
\end{itemize}

\section{Особливості реалізації}
\quad 
Програма використовує бібліотеку num-bigint для роботи з великими числами та реалізує наступні ключові функції:
\begin{itemize}
    \item generate\_random\_prime - генерація простих чисел заданого розміру
    \item mod\_pow - ефективне модульне піднесення до степеня
    \item mod\_inverse - пошук мультиплікативного оберненого
    \item miller\_rabin\_test - тест простоти Міллера-Рабіна
\end{itemize}

\section{Приклад роботи програми}
\quad 
При створенні нової пари ключів з розміром 1024 біти:
\begin{verbatim}
let rsa = RSA::new(1024, 20);
let message = BigInt::from(12345);
let (encrypted_key, signature) = rsa.send_key(&message, 
    &rsa.public_key_e, &rsa.public_key_n);
\end{verbatim}

\section{Результати тестування}

\subsection{Згенеровані ключі}
\subsubsection{Ключі Аліси}
\begin{verbatim}
Публічна експонента (E):
10001

Публічний модуль (N):
1BE32B781E2188EC9A83F61062AF5695535732EF2F628B9E25001C7FAE4A5B7B
47EE78EC6FE22E9E450166297B51A2DE88109D0C9C682DCB4768AE89FADEDF49

Приватний ключ (D):
A9108E77A833A5E6C2D940EA155CE78C61B44315CC2FA23F1E118EB481EE48A8
26AE77CCA86B5E18C76A643F045CDCD385B906E19E3906531BB3068553B801
\end{verbatim}

\subsubsection{Ключі Боба}
\begin{verbatim}
Публічна експонента (E):
10001

Публічний модуль (N):
5CBDEFD0905EBFBE3DBB5F0C25629991DFA666732B97ED71FA2BA5AC71509633
F81E709FE3C402F5B71072B331BA7A54A7CBC6409C37326623B23C716EFA3805

Приватний ключ (D):
2F16D1B95854838F18D116859845E42BC91F384DAA08CE2B5D16E45C167D1086
54CAC09D634E31403107F44773567F3FB233DFD8AACF5474E6D8B667F4066E01
\end{verbatim}

\subsection{Тестування шифрування}
\begin{verbatim}
Оригінальне повідомлення: 48656C6C6F20426F6221
Зашифроване повідомлення: 
5692720C48935A47A2E1EF8AFE46138BDDA24D6540F9D6498D6ECF7B033DFF1A
D12CB01FE943A141A95B187FB42B97DF0D57922891D9E748A02F43274083CAEE
Розшифроване повідомлення: 48656C6C6F20426F6221
\end{verbatim}

\subsection{Тестування цифрового підпису}
\begin{verbatim}
Секретне повідомлення: 
5365637265742066726F6D20416C69636521

Підпис Аліси:
13E3F8DDB6640592B41FF233F3FC7A91871B47FEC7079117630AEC531516F53C
09AD5F32E85095F1848B2251803FD3B7A0EAEAB921977301AB7CE6A0A05E59DE

Зашифроване повідомлення:
B6DB06EF2396F8CACC3303E0D026120079320BC972E3CFC98F357AC9BC711C7B
179C21266C60544756F18A8E2578734FD1062A03C13E73B33F9C6CEDB2C675C

Результат верифікації: Успішно
\end{verbatim}

\subsection{Тестування обміну ключами}
\begin{verbatim}
Сесійний ключ: DEADBEEF
Зашифрований ключ:
1fc8242f1744d07e10b946065135e8f6890587e3b979f49cfc5bb8bc18be2a47
ebb64ec6babdea0552ef3bb9bf2c16e9e4b065f03089dcbb5d96e5e80696ffb4

Зашифрований підпис:
3a31089c88ce4f6be4384759612aa882b4e824e1e2200a27b64f452c6efe82d3
691cbd86d0a24526a20aeb5936a3c33d45ab6981cbb42aef1034b153ee811f21

Результат обміну: Успішно
\end{verbatim}

\section{Висновок}
\quad
У роботі було реалізовано криптосистему RSA з використанням сучасних криптографічних методів. Система включає всі необхідні компоненти для безпечного обміну повідомленнями та створення цифрових підписів. Особлива увага була приділена ефективній реалізації основних криптографічних операцій.

\end{document}