\documentclass[12pt]{article}
\usepackage[english,ukrainian]{babel}
\usepackage[letterpaper,top=2cm,bottom=2cm,left=3cm,right=3cm,marginparwidth=1.75cm]{geometry}
\usepackage{amsmath, booktabs, xcolor, hyperref}

\title{\textbf{Аналіз генераторів псевдовипадкових послідовностей}}
\author{}
\date{}

\begin{document}
\maketitle

\section{Мета роботи}
Вивчення критеріїв згоди і набуття навичок у побудові та застосуванні тестів для перевірки статистичних властивостей бінарних випадкових і псевдовипадкових послідовностей, ознайомлення з поняттям М-послідовності.

\section{Теоретичні відомості}
У роботі розглядаються наступні генератори:
\begin{itemize}
\item Вбудований генератор Rust
\item Генератор Лемера
\item L20 та L89 генератори
\item Генератор Джеффі
\item Генератор Вольфрама
\item BM генератор
\item BBS генератор
\end{itemize}

Для оцінки якості генераторів використовуються три статистичні тести:
\begin{itemize}
\item Тест на рівномірність розподілу
\item Тест на незалежність
\item Тест на однорідність
\end{itemize}

\section{Практична частина}
\subsection{Хід роботи}
Для кожного генератора було згенеровано послідовність довжиною 1 000 000 байт. Тести проводились для трьох рівнів значущості: 0.01, 0.05 та 0.10.
\newpage
\subsection{Результати}
\begin{table}[htbp]
\centering
\small
\begin{tabular}{l*{9}{c}}
\toprule
\multirow{}{}{\textbf{Генератор}} & \multicolumn{3}{c}{\textbf{$\alpha = 0.01$}} & \multicolumn{3}{c}{\textbf{$\alpha = 0.05$}} & \multicolumn{3}{c}{\textbf{$\alpha = 0.10$}} \\
\cmidrule(lr){2-4} \cmidrule(lr){5-7} \cmidrule(lr){8-10}
 & \textbf{U} & \textbf{I} & \textbf{H} & \textbf{U} & \textbf{I} & \textbf{H} & \textbf{U} & \textbf{I} & \textbf{H} \\
\midrule
Вбудований (біти)  & \textcolor{green}{+} & \textcolor{green}{+} & \textcolor{green}{+} & \textcolor{green}{+} & \textcolor{green}{+} & \textcolor{green}{+} & \textcolor{red}{-} & \textcolor{green}{+} & \textcolor{green}{+} \\
Вбудований (байти) & \textcolor{green}{+} & \textcolor{green}{+} & \textcolor{green}{+} & \textcolor{red}{-} & \textcolor{green}{+} & \textcolor{green}{+} & \textcolor{green}{+} & \textcolor{red}{-} & \textcolor{green}{+} \\
Лемер (низькі)     & \textcolor{green}{+} & \textcolor{red}{-} & \textcolor{green}{+} & \textcolor{green}{+} & \textcolor{red}{-} & \textcolor{green}{+} & \textcolor{green}{+} & \textcolor{red}{-} & \textcolor{green}{+} \\
Лемер (високі)     & \textcolor{red}{-} & \textcolor{green}{+} & \textcolor{green}{+} & \textcolor{red}{-} & \textcolor{green}{+} & \textcolor{green}{+} & \textcolor{red}{-} & \textcolor{green}{+} & \textcolor{green}{+} \\
L20 (1M біт)       & \textcolor{green}{+} & \textcolor{green}{+} & \textcolor{green}{+} & \textcolor{green}{+} & \textcolor{green}{+} & \textcolor{green}{+} & \textcolor{green}{+} & \textcolor{green}{+} & \textcolor{green}{+} \\
L20 (16M біт)      & \textcolor{green}{+} & \textcolor{green}{+} & \textcolor{green}{+} & \textcolor{green}{+} & \textcolor{green}{+} & \textcolor{green}{+} & \textcolor{green}{+} & \textcolor{green}{+} & \textcolor{green}{+} \\
L89                & \textcolor{green}{+} & \textcolor{green}{+} & \textcolor{green}{+} & \textcolor{green}{+} & \textcolor{green}{+} & \textcolor{green}{+} & \textcolor{red}{-} & \textcolor{green}{+} & \textcolor{green}{+} \\
Джеффі             & \textcolor{green}{+} & \textcolor{red}{-} & \textcolor{green}{+} & \textcolor{green}{+} & \textcolor{red}{-} & \textcolor{green}{+} & \textcolor{green}{+} & \textcolor{red}{-} & \textcolor{green}{+} \\
Вольфрам           & \textcolor{green}{+} & \textcolor{red}{-} & \textcolor{green}{+} & \textcolor{red}{-} & \textcolor{red}{-} & \textcolor{green}{+} & \textcolor{red}{-} & \textcolor{red}{-} & \textcolor{green}{+} \\
Бібліотекар         & \textcolor{red}{-} & \textcolor{red}{-} & \textcolor{green}{+} & \textcolor{red}{-} & \textcolor{red}{-} & \textcolor{green}{+} & \textcolor{red}{-} & \textcolor{red}{-} & \textcolor{green}{+} \\
BM (біти)          & \textcolor{green}{+} & \textcolor{green}{+} & \textcolor{green}{+} & \textcolor{green}{+} & \textcolor{green}{+} & \textcolor{green}{+} & \textcolor{green}{+} & \textcolor{green}{+} & \textcolor{green}{+} \\
BM (байти)         & \textcolor{green}{+} & \textcolor{green}{+} & \textcolor{green}{+} & \textcolor{green}{+} & \textcolor{red}{-} & \textcolor{green}{+} & \textcolor{green}{+} & \textcolor{green}{+} & \textcolor{green}{+} \\
BBS (біти)         & \textcolor{green}{+} & \textcolor{green}{+} & \textcolor{green}{+} & \textcolor{green}{+} & \textcolor{green}{+} & \textcolor{green}{+} & \textcolor{green}{+} & \textcolor{green}{+} & \textcolor{green}{+} \\
BBS (байти)        & \textcolor{green}{+} & \textcolor{green}{+} & \textcolor{green}{+} & \textcolor{green}{+} & \textcolor{green}{+} & \textcolor{green}{+} & \textcolor{green}{+} & \textcolor{green}{+} & \textcolor{green}{+} \\
\bottomrule
\end{tabular}
\caption{Результати тестів (U - рівномірність, I - незалежність, H - однорідність)}
\label{tab:results}
\end{table}

\subsection{Аналіз результатів}
\begin{itemize}
\item Вбудований генератор Rust показав хороші результати, особливо для бітових послідовностей.
\item Генератор Лемера має проблеми з незалежністю для низьких бітів та рівномірністю для високих бітів.
\item L20 та L89 генератори показали відмінні результати для всіх тестів.
\item Генератор Джеффі має проблеми з незалежністю.
\item Генератор Вольфрама показав найгірші результати, не пройшовши тести на рівномірність та незалежність при α=0.05α=0.05 та α=0.10α=0.10.
\item BM та BBS генератори показали хороші результати, особливо для байтових послідовностей.
\end{itemize}

\section{Висновки}
На основі проведених тестів можна зробити наступні висновки:
\begin{enumerate}
\item Найкращі результати показали L20, L89, BM (байти) та BBS генератори.
\item Генератор Вольфрама та Бібліотекаря виявилия найменш надійним серед досліджених.
\item Вбудований генератор Rust показав хороші результати і може бути рекомендований для загального використання.
\end{enumerate}

Ці результати демонструють важливість вибору відповідного генератора псевдовипадкових послідовностей залежно від конкретних вимог до якості та швидкодії в різних застосуваннях.

\end{document}