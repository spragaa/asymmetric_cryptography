\documentclass[12pt]{article}
\usepackage[english,ukrainian]{babel}
\usepackage[letterpaper,top=2cm,bottom=2cm,left=3cm,right=3cm,marginparwidth=1.75cm]{geometry}
\usepackage{amsmath, booktabs, xcolor, hyperref}

\title{\textbf{Аналіз генераторів псевдовипадкових послідовностей}}
\author{}
\date{}

\begin{document}
\maketitle

\section{Мета роботи}
Вивчення критеріїв згоди і набуття навичок у побудові та застосуванні тестів для перевірки статистичних властивостей бінарних випадкових і псевдовипадкових послідовностей, ознайомлення з поняттям М-послідовності.

\section{Теоретичні відомості}
У роботі розглядаються наступні генератори:
\begin{itemize}
\item Вбудований генератор Rust
\item Генератор Лемера
\item L20 та L89 генератори
\item Генератор Джеффі
\item Генератор Вольфрама
\item BM генератор
\item BBS генератор
\end{itemize}

Для оцінки якості генераторів використовуються три статистичні тести:
\begin{itemize}
\item Тест на рівномірність розподілу
\item Тест на незалежність
\item Тест на однорідність
\end{itemize}

\section{Практична частина}
\subsection{Хід роботи}
Для кожного генератора було згенеровано послідовність довжиною 1 000 000 байт. Тести проводились для трьох рівнів значущості: 0.01, 0.05 та 0.10.
\newpage
\subsection{Результати}
\begin{table}[htbp]
\centering
\small
\begin{tabular}{l*{9}{c}}
\toprule
\multirow{}{}{\textbf{Генератор}} & \multicolumn{3}{c}{\textbf{$\alpha = 0.01$}} & \multicolumn{3}{c}{\textbf{$\alpha = 0.05$}} & \multicolumn{3}{c}{\textbf{$\alpha = 0.10$}} \\
\cmidrule(lr){2-4} \cmidrule(lr){5-7} \cmidrule(lr){8-10}
 & \textbf{U} & \textbf{I} & \textbf{H} & \textbf{U} & \textbf{I} & \textbf{H} & \textbf{U} & \textbf{I} & \textbf{H} \\
\midrule
Вбудований (біти)  & \textcolor{green}{+} & \textcolor{green}{+} & \textcolor{green}{+} & \textcolor{green}{+} & \textcolor{green}{+} & \textcolor{green}{+} & \textcolor{red}{-} & \textcolor{green}{+} & \textcolor{green}{+} \\
Вбудований (байти) & \textcolor{green}{+} & \textcolor{green}{+} & \textcolor{green}{+} & \textcolor{red}{-} & \textcolor{green}{+} & \textcolor{green}{+} & \textcolor{green}{+} & \textcolor{red}{-} & \textcolor{green}{+} \\
Лемер (низькі)     & \textcolor{green}{+} & \textcolor{red}{-} & \textcolor{green}{+} & \textcolor{green}{+} & \textcolor{red}{-} & \textcolor{green}{+} & \textcolor{green}{+} & \textcolor{red}{-} & \textcolor{green}{+} \\
Лемер (високі)     & \textcolor{green}{+} & \textcolor{green}{+} & \textcolor{green}{+} & \textcolor{green}{+} & \textcolor{green}{+} & \textcolor{green}{+} & \textcolor{green}{+} & \textcolor{green}{+} & \textcolor{green}{+} \\
L20 (1M біт)       & \textcolor{green}{+} & \textcolor{green}{+} & \textcolor{green}{+} & \textcolor{green}{+} & \textcolor{green}{+} & \textcolor{green}{+} & \textcolor{green}{+} & \textcolor{green}{+} & \textcolor{green}{+} \\
L20 (16M біт)      & \textcolor{green}{+} & \textcolor{green}{+} & \textcolor{green}{+} & \textcolor{green}{+} & \textcolor{green}{+} & \textcolor{green}{+} & \textcolor{green}{+} & \textcolor{green}{+} & \textcolor{green}{+} \\
L89                & \textcolor{green}{+} & \textcolor{green}{+} & \textcolor{green}{+} & \textcolor{green}{+} & \textcolor{green}{+} & \textcolor{green}{+} & \textcolor{red}{-} & \textcolor{green}{+} & \textcolor{green}{+} \\
Джеффі             & \textcolor{green}{+} & \textcolor{red}{-} & \textcolor{green}{+} & \textcolor{green}{+} & \textcolor{red}{-} & \textcolor{green}{+} & \textcolor{green}{+} & \textcolor{red}{-} & \textcolor{green}{+} \\
Вольфрам           & \textcolor{green}{+} & \textcolor{red}{-} & \textcolor{green}{+} & \textcolor{red}{-} & \textcolor{red}{-} & \textcolor{green}{+} & \textcolor{red}{-} & \textcolor{red}{-} & \textcolor{green}{+} \\
Бібліотекар         & \textcolor{red}{-} & \textcolor{red}{-} & \textcolor{green}{+} & \textcolor{red}{-} & \textcolor{red}{-} & \textcolor{green}{+} & \textcolor{red}{-} & \textcolor{red}{-} & \textcolor{green}{+} \\
BM (біти)          & \textcolor{green}{+} & \textcolor{green}{+} & \textcolor{green}{+} & \textcolor{green}{+} & \textcolor{green}{+} & \textcolor{green}{+} & \textcolor{green}{+} & \textcolor{green}{+} & \textcolor{green}{+} \\
BM (байти)         & \textcolor{green}{+} & \textcolor{green}{+} & \textcolor{green}{+} & \textcolor{green}{+} & \textcolor{red}{-} & \textcolor{green}{+} & \textcolor{green}{+} & \textcolor{green}{+} & \textcolor{green}{+} \\
BBS (біти)         & \textcolor{green}{+} & \textcolor{green}{+} & \textcolor{green}{+} & \textcolor{green}{+} & \textcolor{green}{+} & \textcolor{green}{+} & \textcolor{green}{+} & \textcolor{green}{+} & \textcolor{green}{+} \\
BBS (байти)        & \textcolor{green}{+} & \textcolor{green}{+} & \textcolor{green}{+} & \textcolor{green}{+} & \textcolor{green}{+} & \textcolor{green}{+} & \textcolor{green}{+} & \textcolor{green}{+} & \textcolor{green}{+} \\
\bottomrule
\end{tabular}
\caption{Результати тестів (U - рівномірність, I - незалежність, H - однорідність)}
\label{tab:results}
\end{table}

\section{Значення статистик}

\subsection{Значення статистик для $\alpha = 0.01$}
\begin{table}[htbp]
\centering
\small
\begin{tabular}{l*{3}{c}}
\toprule
\textbf{Генератор} & \textbf{U} & \textbf{I} & \textbf{H} \\
\midrule
Вбудований (біти)  & 221.37 & 64432.64 & 62660.04 \\
Вбудований (байти) & 239.28 & 64796.47 & 58548.00 \\
Лемер (низькі)     & 0.0123 & 2550000.00 & 959.56 \\
Лемер (високі)     & 27.07 & 63304.23 & 59227.07 \\
L20 (1M біт)       & 240.16 & 57572.66 & 63236.65 \\
L20 (16M біт)      & 12.95 & 2859.93 & 55952.54 \\
L89                & 239.23 & 64737.19 & 63002.95 \\
Джеффі             & 257.21 & 80365.19 & 62155.92 \\
Вольфрам           & 309.35 & 81381.06 & 63682.43 \\
BM (біти)          & 259.26 & 65312.66 & 63680.92 \\
BM (байти)         & 221.85 & 65191.36 & 59403.55 \\
BBS (біти)         & 246.22 & 64547.96 & 63137.08 \\
BBS (байти)        & 264.93 & 64885.39 & 58738.91 \\
\bottomrule
\end{tabular}
\caption{Значення статистик для генераторів при $\alpha = 0.01$}
\label{tab:statistics_alpha_01}
\end{table}

\newpage
\subsection{Значення статистик для $\alpha = 0.05$}
\begin{table}[htbp]
\centering
\small
\begin{tabular}{l*{3}{c}}
\toprule
\textbf{Генератор} & \textbf{U} & \textbf{I} & \textbf{H} \\
\midrule
Вбудований (біти)  & 244.08 & 65444.67 & 63079.25 \\
Вбудований (байти) & 256.22 & 64420.84 & 59138.76 \\
Лемер (низькі)     & 0.0123 & 2550000.00 & 959.56 \\
Лемер (високі)     & 27.07 & 63304.23 & 59227.07 \\
L20 (1M біт)       & 224.40 & 57615.74 & 62809.68 \\
L20 (16M біт)      & 10.71 & 2866.93 & 56273.07 \\
L89                & 206.78 & 65160.20 & 63324.13 \\
Джеффі             & 251.96 & 80899.85 & 61608.33 \\
Вольфрам           & 309.35 & 81381.06 & 63682.43 \\
BM (біти)          & 258.58 & 64830.99 & 62489.01 \\
BM (байти)         & 237.31 & 65369.33 & 57282.63 \\
BBS (біти)         & 273.79 & 64917.10 & 63664.18 \\
BBS (байти)        & 268.28 & 65380.38 & 59009.51 \\
\bottomrule
\end{tabular}
\caption{Значення статистик для генераторів при $\alpha = 0.05$}
\label{tab:statistics_alpha_05}
\end{table}

\subsection{Значення статистик для $\alpha = 0.10$}
\begin{table}[htbp]
\centering
\small
\begin{tabular}{l*{3}{c}}
\toprule
\textbf{Генератор} & \textbf{U} & \textbf{I} & \textbf{H} \\
\midrule
Вбудований (біти)  & 267.82 & 64656.59 & 63456.49 \\
Вбудований (байти) & 246.06 & 65121.71 & 59667.63 \\
Лемер (низькі)     & 0.0123 & 2550000.00 & 959.56 \\
Лемер (високі)     & 27.07 & 63304.23 & 59227.07 \\
L20 (1M біт)       & 215.20 & 57291.03 & 63294.49 \\
L20 (16M біт)      & 10.82 & 2870.24 & 55772.82 \\
L89                & 246.26 & 65281.42 & 62700.88 \\
Джеффі             & 238.68 & 80678.48 & 61826.21 \\
Вольфрам           & 309.35 & 81381.06 & 63682.43 \\
BM (біти)          & 258.64 & 64583.78 & 62918.33 \\
BM (байти)         & 261.51 & 65415.49 & 59968.80 \\
BBS (біти)         & 255.20 & 64894.51 & 64034.86 \\
BBS (байти)        & 246.69 & 65380.38 & 59009.51 \\
\bottomrule
\end{tabular}
\caption{Значення статистик для генераторів при $\alpha = 0.10$}
\label{tab:statistics_alpha_10}
\end{table}

\subsection{Аналіз результатів}
\begin{itemize}
\item Вбудований генератор Rust показав хороші результати, особливо для бітових послідовностей.
\item Генератор Лемера має проблеми з незалежністю для низьких бітів та рівномірністю для високих бітів.
\item L20 та L89 генератори показали відмінні результати для всіх тестів.
\item Генератор Джеффі має проблеми з незалежністю.
\item Генератор Вольфрама показав найгірші результати, не пройшовши тести на рівномірність та незалежність при α=0.05α=0.05 та α=0.10α=0.10.
\item BM та BBS генератори показали хороші результати, особливо для байтових послідовностей.
\item Генератори L20 (1M bit) та L20 (16M bit) проходять тест на рівноімовірність, хоча значення статистик суттєво відрізняються, для 16М біт це значення близько 10, для 1М воно близьке до 260, це пов'язано зі структурою побудови лінійного регістру зсуву
\item Вбудований генератор мови Rust використовує OsRng для генерації випадкових бітів і байтів, беручи ентропію з джерел операційної системи.
\end{itemize}

\section{Висновки}
На основі проведених тестів можна зробити наступні висновки:
\begin{enumerate}
\item Найкращі результати показали L20, L89, BM (байти) та BBS генератори.
\item Генератор Вольфрама та Бібліотекаря виявилия найменш надійним серед досліджених.
\item Вбудований генератор Rust показав хороші результати і може бути рекомендований для загального використання.
\end{enumerate}

Ці результати демонструють важливість вибору відповідного генератора псевдовипадкових послідовностей залежно від конкретних вимог до якості та швидкодії в різних застосуваннях.

\end{document}